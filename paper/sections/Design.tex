In this section, we show the design of our system. Additionally, we discuss some
of the challenges, limitations, and design decisions that went into making it.
Finally, we discuss some of the details of the implementation.

\subsection{Gathering \reddit User Names} % (fold)
\label{sub:gathering_reddit_user_names}

The first limitation of the \reddit API is that user names are an `open secret'.
If one has the user name of an account, public information about that account
can be retrieved, but there is no way to directly get user names. Instead, what
we were forced to do was scrape the posts of popular Subreddits and get the user
names of the author of every post and comment. In doing so, we were able to
collect over 150K user names. Of the user names we collected, we randomly
selected around 2K to fully gather data on and run our regression model on.
\\
One issue with this approach is that this makes it impossible to identify non-
participants. If a user never comments on a post, or posts a post themselves,
there is no way to know that that user exists. This is unfortunately an
insurmountable limitation. Instead, we chose only to find good and bad users to
train our classifier, and ignore non-participants.

% subsection gathering_reddit_user_names (end)

\subsection{\reddit API Limits} % (fold)
\label{sub:reddit_api_limits}

\reddit has an API limit of 30 requests per minute. We discovered this limit is
not strictly enforced, but in order to be good citizens and as to not get our
access revoked, we knew we had to design around this constraint. In order to
speed up our ability to access user data (as well as change the features that we
used; see \Cref{sub:picking_user_features}, we crawled user data and put the
raw, unmodified data into a MongoDB instance. This MongoDB served as a cache for
the system. Not were we able to store raw data from \reddit API calls, we were
also able to cache results from more computationally intensive features.

% subsection reddit_api_limits (end)

\subsection{Establishing A Ground Truth} % (fold)
\label{sub:establishing_a_ground_truth}
% will add more detail later. take a look at it, i wrote this at 2 am.
Establishing ground truth was done in two stages. The first stage was to find
reliable users. Finding these users was trivial as the site rewards positive behavior
through karma, which leads to increased visibility. From there the users could
be filtered by their level of contribution manually. Moderators from various
communities were also taken for their work in helping the community. These users
were used in our training set for a +1 reliability score. \\
The second stage was to find users who were unreliable or detrimental to the community.
Eventually, we discovered subreddits dedicated to weeding out users that didn't
contribute and various posts that detailed accounts that were used to abuse the
community. These were used as our training set for a -1 reliability score.

% subsection establishing_a_ground_truth (end)

\subsection{Picking User Features: Exploratory Data Analysis} % (fold)
\label{sub:picking_user_features}

\[ TODO \]

% subsection picking_user_features (end)

\begin{table}[tb]
    \caption{The features used to create the regression model}
    \label{tab:tablename}
    \centering
    \scriptsize

    \begin{tabular}{l|l}
    \hline

    \hline
    \textbf{Feature Description} & \textbf{Importance \%} \\
    \hline
Is Reddit Gold                               & 0.13141645  \\
Has Verified Email                           & 0.75193818  \\
Time Account Created                         & 0.36501668  \\
Flesch--Kincaid Readability of Comments      & 0.1780538   \\
Link Karma                                   & 51.24893713 \\
Number of Gilded Posts                       & 0.          \\
Number of Total Posts                        & 9.34206046  \\
Average Karma per Post                       & 13.51664424 \\
\% of Posts Gilded                           & 0.          \\
Comment Karma                                & 21.0226671  \\
Number of Gilded Comments                    & 0.          \\
Average Comment Karma per Comment            & 0.49150516  \\
\% of Comments Gilded                        & 0.          \\
\% of Post Karma - Trusted Subreddits        & 0.04919857  \\
\% of Post Karma - Top 100 Subreddits        & 0.11182328  \\
\% of Post Karma - Top 50 Subreddits         & 0.10522454  \\
\% of Post Karma - Top 25 Subreddits         & 0.03107503  \\
\% of Post Karma - Top 10 Subreddits         & 0.02660348  \\
\% of Comment Karma - Trusted Subreddits     & 0.11130672  \\
\% of Comment Karma - Top 100 Subreddits     & 1.7193966   \\
\% of Comment Karma - Top 50 Subreddits      & 0.09484735  \\
\% of Comment Karma - Top 25 Subreddits      & 0.06505774  \\
\% of Comment Karma - Top 10 Subreddits      & 0.34236058  \\
\% of Swear Words Used in Comments           & 0.03435242  \\
Unique Words / Total Number Words            & 0.26051451  \\
    \hline

    \hline
    \end{tabular}
\end{table}


\subsection{Picking a Regression Model} % (fold)
\label{sub:picking_a_regression_model}

Something something neural nets give no insight

something something decision trees

something something random forest

something something darkside

% subsection picking_a_regression_model (end)
