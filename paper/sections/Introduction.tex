\reddit{}, the self-proclaimed `frontpage of the internet`, is slowly becoming
just that; many popular news sources, such as The New York Times and CNN, have
begun to use \reddit{} as a primary source of information in an effort to
leverage the power of the large numbers of users which gather and try to provide
and cross-validate information others have gathered.

In most cases, the information gathered is rather inoccuous. For example, a
subset of \reddit{} users (or \textit{redditors}) gather information from
popular news site and aggregate them on individual, topic-based forums (called
\textit{subreddits}) like \texttt{r/news} or \texttt{r/worldnews}, subreddits
for current events within the United States and the world, respectively.

However, \reddit{}'s track record is not necessarily completely spotless when it
comes to dispersing correct, cross-validated information. For example, during
the Boston bombings of the Boston Marathon in April of 2013, \reddit{} falsely
accused individuals of having been part of the bombings simply because of
circumstantial evidence. Many news sites, including CNN, picked up the story and
publicized it. However, it turned out that the individuals \reddit{} picked out
were completely innocent and were only identified as being potentially involved
because of circumstantial evidence (such as having been missing for long periods
of time).\cite{Potts:2013:IRC:2507065.2507079}.

\reddit{} is also noteworthy for allowing users to \textit{upvote} or
\textit{downvote} user-submitted content, indicating either agreement/interest
or disagreement/disinterest, respectively, with the material at hand. As a side
effect of this, users ultimately promote material that they find to be
interesting to them, and hide material that they find to be out of scope,
irrelevant, or simply not interesting with respect to the subreddit they're
in.\cite{Gilbert:2013:WUR:2441776.2441866} Additionally, every upvote and
downvote is connected with a measurement of the community's net
happiness/satisfaction with the content submitted. A user gains \textit{karma}
for every upvote they receive, and loses karma for every downvote they get. 
