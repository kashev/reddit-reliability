\reddit{}, the online social network which calls itself the `frontpage of the
internet', is slowly becoming just that; many popular news sources, such as The
New York Times and CNN, have begun to use \reddit{} as a primary source of
information in an effort to leverage the power of the large numbers of users
which gather and try to provide and cross-validate information others have
supplied.

In most cases, the information users share on \reddit{} is rather inoccuous. For
example, a subset of \reddit{} users (or \textit{redditors}) gather information
from popular news sites and aggregate them on individual, topic-based forums
(called \textit{subreddits}) like \texttt{r/news} or \texttt{r/worldnews},
subreddits for current events within the United States and the world,
respectively, in an effort to share and discuss events going on in the world
around them.

However, \reddit{}'s track record is not necessarily completely spotless when it
comes to dispersing correct, cross-validated information. For example, during
the Boston bombings of the Boston Marathon in April of 2013, \reddit{} falsely
accused individuals of having been part of the bombings simply because of
circumstantial evidence. Many news sites, including CNN, picked up the story and
publicized it. However, it turned out that the individuals \reddit{} picked out
were completely innocent and were only identified as being potentially involved
because of circumstantial evidence (such as having been missing for long periods
of time).\cite{Potts:2013:IRC:2507065.2507079}.

\reddit{} is also noteworthy for allowing users to \textit{upvote} or
\textit{downvote} user-submitted content, indicating either agreement/interest
or disagreement/disinterest, respectively, with the material at hand. As a side
effect (and as part of the site's algorithm to keep fresh content up longer),
users ultimately promote material that they find to be interesting to them, and
hide material that they find to be out of scope, irrelevant, or simply not
interesting with respect to the subreddit they're
in.\cite{Gilbert:2013:WUR:2441776.2441866} Additionally, every upvote and
downvote is connected with a measurement of the community's net
happiness/satisfaction with the content submitted. A user gains \textit{karma}
for every upvote they receive, and loses karma for every downvote they get.

In this project, we present a system by which we glean more meaning from
redditor data than simple community response; instead, we develop a metric by
which to judge the \textit{reliability} of a user. This realiaiblity score
ranges from $-1 \leq s_r \leq 1$, and is computed from a model trained on a
hand-picked ground truth of reliable and unreliable users.
