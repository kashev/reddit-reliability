Prior to beginning the project it was important to gather and look at related work to determine how the concept stood in terms of current research. Various groups have leveraged social media in the past as a resource for information retrieval. A group that interested us was working on Apollo to do fact finding on Twitter. Part of their analysis involved gathering credibility for claims and sources.\cite{Le:2011:DDL:2070942.2071018} This method proved to be critical for them and could be well used in a more community-based environment.

% Another analysis of user behavior on Twitter observed how the silent majority of users behaved and attempted to extract information from them. \cite{DBLP:journals/corr/TagarelliI15}

Though they were difficult to find, there were some groups that had done research on \reddit{}. A foreboding analysis of the behavior of \reddit{} during the boston marathon bombings encouraged the idea that, while the actions taken by one community (/r/findbostonbombers) were destructive, \reddit{} operates differently than indicated by these events. \cite{Potts:2013:IRC:2507065.2507079}

Another paper that focused on the analysis of one subreddit (/r/sandy) finds that \"perspective-created\" content on \"social news sites\" tends to place above professional content such as reports from CNN. \cite{Leavitt:2014:UHS:2556288.2557140} This is indicative that \"fresh\" information that would not be otherwise available is present on \reddit{} and needs retrieval.

Another paper on voting behaviors was done to analyze the underprovision, or lack of voting participation, on \reddit{}. A surprising statistic to come out of this was that nearly 52\% of popular links were ignored the first time they were introduced.\cite{Gilbert:2013:WUR:2441776.2441866} Possibly indicating a lack of alacrity in terms of information delivery. It is apparent that finding and detecting information is an important and non-trivial task.



