\documentclass[a4paper]{article}

\usepackage[utf8]{inputenc}
\usepackage[T1]{fontenc}
\usepackage[margin=1in]{geometry}
\usepackage{amsmath}
\usepackage{graphicx}

\title{Critique of GruMon Paper}
\author{Kashev Dalmia, Ryan Freedman, Terence Nip \\
        \texttt{\{dalmia3, rtfreed2, nip2\}@illinois.edu}
       }
\date{\today}

\begin{document}

\maketitle

\section{Summary}
The paper summarizes the process that the researchers went through to develop
the GruMon system. After discussing previous attempts to create classifiers with
the same goal the paper describes the method it uses. The system is used to
distinguish individuals from groups by detecting activity/location information
using phone sensors, computing similarities between pairs, and passing them
through an SVM classifier. The motivations behind developing GruMon were
primarily for group detection to allow for better advertising and resource
planning in crowded areas. There were three trials used for measurement and
testing: a Korean mall, a Singaporean mall, and an International Airport. The
three trials had varying means of detection, participants, and incentives for
participation.

\section{Critique}
First, we found it curious that the authors chose to use a control group of
phones for the more ``difficult to navigate'' Mall 2. It is possible that a more
random sample of phones, with more run of the mill specs, wouldn't have been
able to navigate the second mall as well as the Samsung Galaxy S III's. Thus,
their strong precision and recall numbers are possibly inflated.

Second, the training of the classifier is problematic. They stated that they
trained the classifier on ground truth from each mall. They did not state
clearly how the initial classifier for the airport data was trained. Though
their accuracy is quite good, it is only quite good for venues that have been
pre-scouted, by willing participants. If for every new venue GruMon is deployed
to, a contingent of paid participants has to be sent out with a special app on
their phones, using GruMon for useful purposes is going to become extremely
costly and impractical extremely fast.

Finally, though the authors say that there are issues with scaling GruMon, it is
strange that the only data sets that they have with ground truth are $O(100)$,
nor are there any ground truth labels for any data that's not in a mall. Without
knowing the precision for venues with different usage patterns (like airports,
concert venues, larger more open venues like university campuses), which might
be larger than $O(100)$, it is hard to declare this problem solved by GruMon. It
is possible that their classifier only works well in malls. The authors say that
Mall 2's access points were poor and not as useful because of the large atrium
in the mall, but no further detail is given. There is too much variability in
the malls to know for certain what the causes of the discrepancies in precision
are.

\section{Praise}
Despite our criticisms, there several things that the paper did well. At the end
of the day, for both Mall data sets, GruMon was able to achieve over 90\%
precision, which is quite good for the purposes of advertising. Another thing
that they did well was feature selection; though they did use several features
for their classifier, all those features were chosen based on an insight of how
actual humans behave in groups. For instance , the barometer was used and given
high weight because humans in groups tend not to be on different floors of a
venue, even if they're visiting different stores within the floor. It would have
been easy to just log values for every sensor the phone had and throw them in a
classifier, but instead, they chose carefully, and thus saved battery life of
the devices used. Finally, it was interesting and impressive that they had a
good method of combating poor external infrastructure, like the few and poorly
placed WiFi access points in Mall 2 (though we have our concerns, as mentioned
above.)

\end{document}
