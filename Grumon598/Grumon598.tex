\documentclass[12pt]{amsart}

\usepackage{amsmath}    % need for subequations
\usepackage{graphicx}   % need for figures
\usepackage{verbatim}   % useful for program listings
\usepackage{color}      % use if color is used in text
\usepackage{subfigure}  % use for side-by-side figures
\usepackage{hyperref}   % use for hypertext links, including those to external documents and URLs
\usepackage[affil-it]{authblk}
\usepackage[margin = 1in] {geometry}
\usepackage{float}
\usepackage{layouts}
%\usepackage{gensymb}

\makeatletter
\def\@maketitle{
  \newpage
  \null
  \vskip 2em
  \begin{center}
  \let \footnote \thanks
    {\Large\bfseries \@title \par}
    \vskip 1.5em
    {\normalsize
      \lineskip .5em
      \begin{tabular}[t]{c}
        \@author
      \end{tabular}\par}
    \vskip 1em
    {\normalsize \@date}
  \end{center}
  \par
  \vskip 1.5em}
\makeatother


\title{Critique of GruMon: Fast and Accurate Group Monitoring for Heterogeneous Urban Spaces}
\author[]{Kashev Dalmia, Ryan Freedman, and Terence Nip}
\affil[]{University of Illinois at Urbana-Champaign}
\date{\today}

\raggedbottom

\begin{document}

\maketitle

\section{\bf Summary}
The paper summarizes the process that the researchers went through to develop
the GruMon system. After discussing previous attempts to create classifiers with
the same goal the paper describes the method it uses. The system is used to
distinguish individuals from groups by detecting activity/location information
using phone sensors, computing similarities between pairs, and passing them
through an SVM classifier. The motivations behind developing GruMon were
primarily for group detection to allow for better advertising and resource
planning in crowded areas. There were three trials used for measurement and
testing: a Korean mall, a Singaporean mall, and an International Airport. The
three trials had varying means of detection, participants, and incentives for
participation.

\section{\bf Critique}
\begin{center}
% These are just bullets now, will correct later
- Airport didn't have ground truth group number, cannot apply accuracy at any other scale than O(100) \\
- Device fragmentation (i.e. h/w differences) + poor signals = wat do? \\
- BLE increasingly viable bc smartwatches (bluetooth usually on) - even back in the ancient days of 2014 \\
- Weird choice to hand out phones for one venue but allow use of personal phones at others; introduces bias towards potential hardware pros/cons whereas that bias is nonexistent for other venues \\
- tl;dr becomes an engineering problem at the end of the day \\
- They make mention of the fact that if there are folks in a party who aren't sending signals, they're useless (i.e. partial party) - what's the motivation to get folks to share their signals? (You'd have to already have a good sense of what your "shopper profile" is like prior to doing anything...)\\
- Interesting that they punted on locked down OSes after mentioning anon traces -- something something wifi triangulation? (Is it the case that access points just broadcast and it's up to the end user device to ack \& request connection, or is it the case that the end user device polls the AP?) \\
\end{center}

\section{\bf Praise}
Despite our criticisms, there several things that the paper did well. At the end
of the day, for both Mall data sets, GruMon was able to achieve over 90\%
precision, which is quite good for the purposes of advertising. Another thing
that they did well was feature selection; though they did use several features
for their classifier, all those features were chosen based on an insight of how
actual humans behave in groups. For instance, the barometer was used and given
high weight because humans in groups tend not to be on different floors of a
venue, even if they're visiting different stores within the floor. It would have
been easy to just log values for every sensor the phone had and throw them in a
classifier, but instead, they chose carefully, and thus saved battery life of
the devices used. Finally, it was interesting and impressive that they had a
good method of combating poor external infrastructure, like the few and poorly
placed WiFi access points in Mall 2.

\end{document}
