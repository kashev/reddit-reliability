\documentclass[a4paper]{article}

\usepackage[utf8]{inputenc}
\usepackage[T1]{fontenc}
\usepackage[margin=1in]{geometry}
\usepackage{amsmath}
\usepackage{graphicx}

\title{Critique of Instagram Paper}
\author{Kashev Dalmia, Ryan Freedman, Terence Nip \\
        \textit{\{dalmia3, rtfreed2, nip2\}@illinois.edu}
       }
\date{\today}

\begin{document}

\maketitle

\section{Summary}
This paper talked about the relevance of the Instagram in the context of social sensing. Given the context of the instagram as a new technology there exists great potential for it to be used as a sensing medium where, previously, no such thing existed. They explore the metadata associated with the information presented that was not previously available. This was more of a passive act leading to decreased friction in data generation. Because humans don't touch the data, odds are, that the data is more valid than conscious user generated data. 

\section{Critique}

- English, do you speak it? \\
- Paper without a purpose. \\
- Inconsistent terminology quadrants -> sectors \\
- Stating the obvious \\
- Relies on tweeting instagram photos -> biased data. \\

\section{Praise}
- Good job exploring a new medium for social sensing. Set the ground work for future papers.

\end{document}
