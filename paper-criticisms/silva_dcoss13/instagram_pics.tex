\documentclass[a4paper]{article}

\usepackage[utf8]{inputenc}
\usepackage[T1]{fontenc}
\usepackage[margin=1in]{geometry}
\usepackage{amsmath}
\usepackage{graphicx}

\title{Critique of Instagram Paper}
\author{Kashev Dalmia, Ryan Freedman, Terence Nip \\
        \textit{\{dalmia3, rtfreed2, nip2\}@illinois.edu}
       }
\date{\today}

\begin{document}

\maketitle

\section{Summary}
This paper talked about the relevance of the Instagram in the context of social sensing. Given the context of the instagram as a new technology there exists great potential for it to be used as a sensing medium where, previously, no such thing existed. They explore the metadata associated with the information presented that was not previously available. This was more of a passive act leading to decreased friction in data generation. Because humans don't touch the data, odds are, that the data is more valid than conscious user generated data. 

\section{Critique}

The first and most obvious critique of the paper is the presentation of the paper itself. The writing made it difficult to access the information presented. Overuse of certain words ('planetary' in particular) and a lack of proof-reading made the paper seem very disconnected (referring to Instagram as "The Instagram"). A related issue was a lack of consistency in their terminology particularly in their breakdown of location (see quadrants vs sectors). \\
A second critique of the paper is that they had no purpose or motivation going in. The entrance of Instagram as a participatory sensing system, while important to catalogue, should not be the exclusive focus of a research paper.\\
They purposefully attempt to obfuscate the contents of their paper despite the trivial nature its content.\\
- Relies on tweeting instagram photos -> biased data. \\
- Lack of understanding of the motivation of medium \\

\section{Praise}
- Good job exploring a new medium for social sensing. Set the ground work for future papers.

\end{document}
