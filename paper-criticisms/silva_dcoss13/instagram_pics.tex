\documentclass[a4paper]{article}

\usepackage[utf8]{inputenc}
\usepackage[T1]{fontenc}
\usepackage[margin=1in]{geometry}
\usepackage{amsmath}
\usepackage{graphicx}
\usepackage{csquotes}

\title{Critique of Instagram Paper}
\author{Kashev Dalmia, Ryan Freedman, Terence Nip \\
        \textit{\{dalmia3, rtfreed2, nip2\}@illinois.edu}
       }
\date{\today}

\begin{document}

\maketitle

\section{Summary}

This paper talked about the relevance of the Instagram in the context of social
sensing. Given the context of the Instagram as a new technology there exists
great potential for it to be used as a sensing medium where, previously, no such
thing existed. They explore the meta-data associated with the information
presented that was not previously available. This was more of a passive act
leading to decreased friction in data generation. Because humans don't touch the
data, odds are, that the data is more valid than conscious user generated data.

Furthermore, the researchers investigate the data and find some curious trends.
For example, they discover a difference in the temporal posting behaviors of
Brazilians compared to the rest of the world, and they discover a difference in
the treatment of Instagram sharing compared to that of location sharing;
specifically, people are more inclined to share their location `at breakfast',
whereas there is no such spike in Instagram activity. They make attempts to
explain these oddities in their data.

\section{Critique}

The first and most obvious critique of the paper is the presentation of the
paper itself. The writing made it difficult to access the information presented.
Overuse of certain words (`planetary' in particular) and a lack of proof-reading
made the paper seem very disconnected (referring to Instagram as ``The
Instagram''). A related issue was a lack of consistency in their terminology
particularly in their breakdown of location (see quadrants vs. sectors), which
made their particular insights difficult to grasp.

A second critique of the paper is that they had no purpose or motivation going
in. The entrance of Instagram as a participatory sensing system, while important
to catalog, should not be the exclusive focus of a research paper. They found
a data dump, and attempted to glean some insights from it, but on the whole
didn't achieve much insight. They even made no attempt to discover the reason
for some of the discrepancies in their data. For instance, they suggest that
perhaps ``an unusual event may have happened on Tuesday that result in an
abnormal number of shared photos'', but make no effort to discover what this
might have been to give their claim grounds in reality.

Part of their difficulty in gleaning human insight in the usage patterns of
Instagram was their misunderstanding of the motivations of the usage of the
platform. For instance, they state the following regarding Instagram as a PSN:
\begin{displayquote}
    If a particular application requires a more comprehensive coverage, it is
    necessary to encourage users to participate in places they normally would
    not. Micro- payments or scoring systems are examples of alternatives that
    might work in this case.
\end{displayquote}
We disagree that either of these things might work. Instagram gained popularity
by positioning itself as a platform for sharing high quality photos. People post
photos that they fancy artistic, and it is unlikely that they will add content
that they find unsuitable to their taste, no matter what the incentive. People
want to post filtered pictures of their food. They do not want to post pictures
of supermarkets. In a small interview about a study on ReadWrite
(\texttt{\scriptsize\detokenize{
http://readwrite.com/2012/03/14/study_why_do_people_use_instagram}}), Zachary
McCune argues the same; people regard their Instagram feeds as streams of
content curated by themselves, and do so for six main reasons: ``sharing,
documentation, seeing, community, creativity and therapy''. We feel that it is
shortsighted to think that people could be cajoled into using Instagram for pure
sensing purposes if it does nothing to add to their carefully crafted social
media persona.

Finally, we take issue with the way that the data was collected. They state that
they collected geotagged Instagram photos that were also cross-posted to
Twitter. Though the Instagram app makes this sort of cross-posting simple and
frictionless, the percentage of users who do this is certainly below 100\% and
introduces some sort of bias in the usage patterns. Furthermore, if the
researchers are tied to Twitter, why not choose pictures that were posted to
Twitter with geotags, regardless of if they came from Instagram at all? They
argue that this is the only way to find Instagram photos, but we disagree. A
simple system which mines user names, and then searches for them for public
photos, and then finds the geotagged ones, would provide a much more unbiased
system for finding the photos than relying on Twitter.

\section{Praise}

Besides having extremely thorough references, the paper does a good job
exploring a new medium for social sensing. These authors very likely set some
ground work for future papers (it has been cited 22 times, according to Google
Scholar.) Furthermore, their identification of Points of Interest, though given
light treatment in the paper, was quite interesting. If given greater treatment,
this, along with some analysis actual analysis of the photo itself, its tags,
and comments, could lead to some interesting work in automatically recognizing
events in the real world. We were disappointed that the authors stopped short of
actually doing this work.

\end{document}
