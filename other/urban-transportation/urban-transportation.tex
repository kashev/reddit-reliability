\documentclass[a4paper]{article}

\usepackage[utf8]{inputenc}
\usepackage[T1]{fontenc}
\usepackage[margin=1in]{geometry}
\usepackage{amsmath}
\usepackage{graphicx}
\usepackage{csquotes}

\title{Summaries of Papers on Urban Transportation}
\author{Kashev Dalmia, Ryan Freedman, Terence Nip \\
        \textit{\{dalmia3, rtfreed2, nip2\}@illinois.edu}
       }
\date{\today}

\begin{document}

\maketitle

\section{Vehicular Networks in Urban Transportation Systems \& MDDV: A Mobility-
Centric Data Dissemination Algorithm for Vehicular Networks}

In these closely related papers, the authors describe their design and test of a
network enabling smart cars. They argue that a significant challenege associated
with achieving useful smart vehicles is to have a strong and reliable networking
system between them. In the second paper, they present MDDV, which they assert
can be that network. They develop the algorithms mostly for Vehicle to Vehicle,
or V2V, systems. They differentiate V2Vs from traditional ad-hoc networks due to
the mobility of nodes, and the possibility to be highly partitioned, among
others. MDDV takes on these challenges with an approach that ``combines
opportunistic forwarding, geographical forwarding, and trajectory based
forwarding'' of information. Furthermore, they tested this system using
prototype hardware in the Atlanta metro area, with promising results.

\section{Overview on Security Approaches in Intelligent Transportation Systems: Searching for hybrid trust establishment solutions for VANETs}
This paper provides a broad overview and analysis of currently proposed solutions to the security network issues in Vehicular Ad-hoc Networks (VANETs). After exploring both centralized and decentralized solutions the paper goes on to demonstrate that a hybrid approach, currently underrepresented, may be vital to developing a sustainable and efficient security architecture for Intelligent Transportation Systems (ITS).

\section{The Performance of a Crowdsourced Transportation Infomration System}
This paper provides an overview of the existing landscape with respect to mass
transit vehicle arrival prediction systems, discussing issues with GPS and
normal linear regression-based predictions. It proceeds to go into analysis on
the benefit and drawbacks of including crowdsourced data to help augment the
predictions generated from both live vehicle data and from historical vehicular
data.

\end{document}
