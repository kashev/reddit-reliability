In light of the issues with Reddit, it is valuable to determine the reliability
of Redditors automatically. We plan to do this by creating a Reliability Score,
ranging beetween $-1.0$ and $1.0$, in which a more positive number denotes a
user who is more reliable, and a negative score denotes a user who is less
reliable. Furthermore, we hope to show that based on indicators used to
calculate this score over time, that we can anticipate the trend of a Redditor's
reliability. This has implications in creating reliable users by predicting
their trajectory of behavior. One could take this trajectory, compare it to a
more `desirable' trajectory, and perhaps even nudge Redditors towards a more
Reliable trajectory.

The following sections discuss some technical challenges we anticipate, and our
plans to overcome them.

\subsection{Creating A Reliability Score}
\label{sub:creating_a_reliability_scorle}
Talk about:
\begin{enumerate}
    \item Relibility Karma vs overall karma
    \item Posting in trusted subreddits
    \item Posting reliable sources
    \item Others
    \item Fusion method
\end{enumerate}
% subsubsection creating_a_reliability_score (end)

\subsection{Gathering Usernames}
\label{sub:gathering_usernames}

Reddit does not have an API for gathering or searching for usernames. It does
however, allow one to see popular posts on particular Subreddits, or in general,
the front page. We plan to mine usernames by looking at popular posts and
comments on those posts, and logging the usernames associated with those
activities. This list of usernames can then be plugged into our reliability
score calculations, and used to train our classifiers for behavior prediciton.

% subsubsection gathering_usernames (end)

\subsection{Verification}
\label{sub:verification}

Verification of reliability is tricky. Though it is labour intensive, we will
have to select a control group of users, view their history manually, decide if
they are `reliable' or not, and then compare that to the score our classifier
gives.

Verification of behavior prediction is less tricky. We can simply use our models
to predict the future behavior of old users, and then compare that to their
actual trajectory.

% subsubsection verification (end)

